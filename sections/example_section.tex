% !TeX spellcheck = en_US
% !TeX encoding = utf8
% !TeX program = pdflatex
% !TeX root = main.tex
% -*- coding:utf-8 mod:LaTeX -*-


\section{Making a standard section}

\begin{frame}\frametitle{YOUR TITLE}
Normal old frame		
\end{frame}

\begin{frame}	
	\begin{block}{Problem Summary}
	Hey
	\end{block}
	
	
	\begin{exampleblock}{Recap}
	Ho
	\end{exampleblock}
	
	\begin{alertblock}{Going forward}
	Let's go
	\end{alertblock}

\end{frame}


\begin{nosidebarframe}
\centering
 If you want more space, use: \texttt{nosidebarframe}
\end{nosidebarframe}






\begin{frame}[containsverbatim]

If you don't want to have the sidebar, you can use:

\verb|\begin{nosidebarframe}|

If you want a nice status report at the end, you can define tasks with a certain progression using:

\verb|\begin{task}{My first task}{20}|

That way, the \texttt{My first task} \textit{task} will be added to the status report at the end with a completion of \text{20\%}

\end{frame}


\begin{task}{My first task}{100}

Not so complicated \todo{Finish this}

\end{task}

\begin{task}{A more complicated task}{20}

This one

\pagebreak

Was more complicated \todo{The list of TODOs disapears when there are no todos}

\pagebreak

And required multiple pages
\end{task}



\begin{frame}[containsverbatim]
	\begin{lstlisting}[language=json,firstnumber=1,basicstyle=\tiny]
{
    "glossary": {
        "title": "example glossary",
		"GlossDiv": {
            "title": "S",
			"GlossList": {
                "GlossEntry": {
                    "ID": "SGML",
					"SortAs": "SGML",
					"GlossTerm": "Standard Generalized Markup Language",
					"Acronym": "SGML",
					"Abbrev": "ISO 8879:1986",
					"GlossDef": {
                        "para": "A meta-markup language, used to create markup languages such as DocBook.",
						"GlossSeeAlso": ["GML", "XML"]
                    },
					"GlossSee": "markup"
                }
            }
        }
    }
}
	\end{lstlisting}		
	\end{frame}

