% !TeX spellcheck = en_US
% !TeX encoding = utf8
% !TeX program = pdflatex
% !TeX root = main.tex
% -*- coding:utf-8 mod:LaTeX -*-


% For pkg being imported elsewhere, specify the options before loading the package
\PassOptionsToPackage{
            colorlinks = true,
            linkcolor = blue, % Also in the side panel sections
			linkbordercolor = red,
			urlcolor  = {blue!90!black},
            citecolor = black,
            anchorcolor = blue
%			pdfborderstyle={/S/U/W 1}% border style will be underline of width 1pt
}{hyperref}

\documentclass[10pt]{beamer} % With pauses
%\documentclass[handout,10pt]{beamer} % No pauses

%\makeatletter
%\def\Hy@colorlink#1{\begingroup\fontshape{bf}\selectfont}%
%\makeatother

% \makeatletter
%\Hy@AtBeginDocument{%
%  \def\@pdfborder{0 0 1}% Overrides border definition set with colorlinks=true
%  \def\@pdfborderstyle{/S/U/W 1}% Overrides border style set with colorlinks=true
%                                % Hyperlink border style will be underline of width 1pt
%}
%\makeatother


%\usepackage[T1]{fontenc} % Messes with the font ...
\usepackage[utf8]{inputenc}
\usepackage[english]{babel}


\usepackage{cleveref}
% For loading images
%\usepackage{graphicx} % Not needed with beamer
\usepackage{grffile}

\usepackage[official]{eurosym} % for the euro symbol \euro{}
\usepackage{etoolbox}% http://ctan.org/pkg/etoolbox
\usepackage{tabularx} % http://ctan.org/pkg/tabularx
\usepackage{listings} % Code listings
%\usepackage{xcolor} % Not needed with beamer
\usepackage{default}
\usepackage{array} % for defining a new column type
\usepackage{varwidth} %for the varwidth minipage environment
\usepackage{xifthen} % Logic
\usepackage{multido}% Logic
\usepackage{multimedia} % For movies
\usepackage{import}
\usepackage{tcolorbox} % Colored boxes
\usepackage{totcount} % Keeps track of the counter TOTALs
\usepackage{caption} % To use \caption*
\usepackage{csvsimple}
\usepackage{amsmath} % For normal text in math mode $...\text{normal text}...$

% colors
\subimport{}{preamble_src/custom_colors.tex}

\newcommand\myshade{85}
\definecolor{mylinkcolor}{rgb}{0.52, 0.09, 0.71}
\definecolor{mycitecolor}{rgb}{0.89, 0.65, 0.13}
\definecolor{myurlcolor}{rgb}{0.08, 0.55, 0.65}

\hypersetup{
  linkcolor  = mylinkcolor!\myshade!black,
  citecolor  = mycitecolor!\myshade!black,
  urlcolor   = myurlcolor!\myshade!black,
  colorlinks = true,
}

% For JSON code snippetscm/100cm * 2
\colorlet{punct}{red!60!black}
\definecolor{background}{HTML}{EEEEEE}
\definecolor{delim}{RGB}{20,105,176}
\colorlet{numb}{magenta!60!black}
\lstdefinelanguage{json}{
	basicstyle=\normalfont\ttfamily,
	numbers=left,
	numberstyle=\scriptsize,
	stepnumber=1,
	numbersep=8pt,
	showstringspaces=false,
	breaklines=true,
	frame=lines,
	backgroundcolor=\color{background},
	literate=
	*{0}{{{\color{numb}0}}}{1}
	{1}{{{\color{numb}1}}}{1}
	{2}{{{\color{numb}2}}}{1}
	{3}{{{\color{numb}3}}}{1}
	{4}{{{\color{numb}4}}}{1}
	{5}{{{\color{numb}5}}}{1}
	{6}{{{\color{numb}6}}}{1}
	{7}{{{\color{numb}7}}}{1}
	{8}{{{\color{numb}8}}}{1}
	{9}{{{\color{numb}9}}}{1}
	{:}{{{\color{punct}{:}}}}{1}
	{,}{{{\color{punct}{,}}}}{1}
	{\{}{{{\color{delim}{\{}}}}{1}
	{\}}{{{\color{delim}{\}}}}}{1}
	{[}{{{\color{delim}{[}}}}{1}
	{]}{{{\color{delim}{]}}}}{1},
}


\graphicspath{{images/}}

% Fill out the different parts of the header
\subimport{}{preamble_src/custom_constructs.tex} % Normal input/include does not work for this
\subimport{}{preamble_src/styles/goettingen_style.tex}
%\subimport{}{preamble_src/styles/metropolis_style.tex}

\usepackage{xpunctuate} % For the latin accronyms
\providecommand{\eg}[0]{e.g\xperiod}
\providecommand{\ie}[0]{i.e\xperiod}

% YOUR INFO
%%%%%%%%%%%%%%%%%%%%%%%%%%%%%%%%%%%%%%%%%%%%%%%%%%%%%%%%%%%%%%%%%%%%%%%%%%%%%%%%%%%%%%%%%%%%%%%%%%%%%%
\title[]{[[[YOUR COOL TITLE]]]}   
\author[]{[[[YOUR NAME]]]} 
\institute[*]{\email{you@yourmail.com}}
%\date{\today} 
\date{} % no date 
%%%%%%%%%%%%%%%%%%%%%%%%%%%%%%%%%%%%%%%%%%%%%%%%%%%%%%%%%%%%%%%%%%%%%%%%%%%%%%%%%%%%%%%%%%%%%%%%%%%%%%	



\begin{document}
	
	% Title frame without side panel and CENTERED	 (+/- depending on right/left side panel position)
	\begin{nosidebarframe}\titlepage\end{nosidebarframe}

    \listOfTodosFrame % Only appears if there actually are TODOs (\todo{})

	% Table of contents
	\frame{\frametitle{Table of contents}{\hypersetup{linkcolor=black}\tableofcontents}} 


	% YOUR SECTIONS
    %%%%%%%%%%%%%%%%%%%%%%%%%%%%%%%%%%%%%%%%%%%%%%%%%%%%%%%%%%%%%%%%%%%%%%%%%%%%%%%%%%%%%%%%%%%%%%%%%%
	% !TeX spellcheck = en_US
% !TeX encoding = utf8
% !TeX program = pdflatex
% !TeX root = main.tex
% -*- coding:utf-8 mod:LaTeX -*-


\section{Making a standard section}

\begin{frame}\frametitle{YOUR TITLE}
Normal old frame		
\end{frame}

\begin{frame}	
	\begin{block}{Problem Summary}
	Hey
	\end{block}
	
	
	\begin{exampleblock}{Recap}
	Ho
	\end{exampleblock}
	
	\begin{alertblock}{Going forward}
	Let's go
	\end{alertblock}

\end{frame}


\begin{nosidebarframe}
\centering
 If you want more space, use: \texttt{nosidebarframe}
\end{nosidebarframe}


\begin{frame}[allowframebreaks]{Long frames}
    You\\ can\\ use\\ \texttt{[allowframebreaks]}\\
    \framebreak
    to\\ split\\ the slides\\
    \framebreak
    as\\ many\\ times\\ as\\ you\\ want\\
\end{frame}


\begin{frame}[containsverbatim]

If you don't want to have the sidebar, you can use:

\verb|\begin{nosidebarframe}|

If you want a nice status report at the end, you can define tasks with a certain progression using:

\verb|\begin{task}{My first task}{20}|

That way, the \texttt{My first task} \textit{task} will be added to the status report at the end with a completion of \text{20\%}

\end{frame}


\begin{task}{My first task}{100}

Not so complicated \todo{Finish this}

\end{task}

\begin{task}{A more complicated task}{20}

This one

\pagebreak

Was more complicated \todo{The list of TODOs disapears when there are no todos}

\pagebreak

And required multiple pages
\end{task}



\begin{frame}[containsverbatim]
	\begin{lstlisting}[language=json,firstnumber=1,basicstyle=\tiny]
{
    "glossary": {
        "title": "example glossary",
		"GlossDiv": {
            "title": "S",
			"GlossList": {
                "GlossEntry": {
                    "ID": "SGML",
					"SortAs": "SGML",
					"GlossTerm": "Standard Generalized Markup Language",
					"Acronym": "SGML",
					"Abbrev": "ISO 8879:1986",
					"GlossDef": {
                        "para": "A meta-markup language, used to create markup languages such as DocBook.",
						"GlossSeeAlso": ["GML", "XML"]
                    },
					"GlossSee": "markup"
                }
            }
        }
    }
}
	\end{lstlisting}		
	\end{frame}


    %%%%%%%%%%%%%%%%%%%%%%%%%%%%%%%%%%%%%%%%%%%%%%%%%%%%%%%%%%%%%%%%%%%%%%%%%%%%%%%%%%%%%%%%%%%%%%%%%%
	
	% Conclusion
	\section{Status report} 

		\completedTable  %Only appears if there are compoleted \begin{tasks}{100} content...\end{tasks}

		\inDevTable %Only appears if there are compoleted \begin{tasks}{5} content...\end{tasks}

	\subsection{Proposals} 
\begin{frame}{\subsecname}

	\begin{block}{Conclusions}
		\begin{itemize}
		\item conclusion 1
		\item conclusion 2
		\item conclusion 3
		\end{itemize}
	\end{block}

	\begin{exampleblock}{Proposals}
		\begin{itemize}
		\item proposal 1
		\item proposal 2
		\item proposal 3
		\end{itemize}
	\end{exampleblock}
\end{frame}



\begin{nosidebarframe}

\centering

\color{blue}
\huge Thank you for your time 

\vskip 2cm

\huge Any questions?

\end{nosidebarframe}

	

\end{document}
